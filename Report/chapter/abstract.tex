\chapter{Abstract}
This machine learning project focused on predicting the quality of red wines based on their chemical properties. To achieve this, several preprocessing techniques were applied, including scaling using different methods and applying log and boxcox transformations. Exploratory data analysis was also performed to better understand the relationships between the features and the target variable.

Several popular machine learning algorithms were then trained and compared, including logistic regression, SVM, random forest, decision trees, and boosting algorithms. Performance metrics such as accuracy, precision, recall, and F1 score were used to identify the best algorithm and preprocessing technique. The random forest model with id = 1 was found to be the most effective, with a micro F1 score of 0.73.

Overall, this project highlights the importance of proper preprocessing techniques and algorithm selection for developing an accurate and effective model for wine quality classification. The findings have practical implications for industries such as wine production, where the ability to predict wine quality could be invaluable.

\textbf{Keywords:} Wine Quality, Data Analysis, Machine Learning, Random Forest, Django






