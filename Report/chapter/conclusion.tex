\chapter{Conclusion}
In conclusion, this project involved exploring various machine learning algorithms for wine quality classification. We also learned several data preprocessing techniques, such as scaling using different scalers, and applying log and boxcox transformations to normalize the data.

After comparing the performance of several models, we found that the random forest model with id = 1 performed the best, with a micro F1 score of 0.73. This suggests that the random forest algorithm is effective for classifying wine quality and can be used in practical applications.

Overall, this project highlights the importance of selecting appropriate preprocessing techniques and evaluating various machine learning algorithms to develop an accurate and effective model for wine quality classification.

